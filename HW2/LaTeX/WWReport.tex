%% ---------------------------------------------------
%% Notice that we use the "report" class instead of "article"
%% ---------------------------------------------------
\documentclass{report}
\usepackage{634format}
\title{Report for ML Exercises }

\author{Wei-Ling Wang}
\date{\today}



%% ---------------------------------------------------
%% enumerate 
%% ---------------------------------------------------
\usepackage{enumerate}

%% ---------------------------------------------------
%% listings is used for including our source code in reports
%% ---------------------------------------------------
\usepackage{listings}
\usepackage{textcomp}

%% ---------------------------------------------------
%% Packages for math environments
%% ---------------------------------------------------
\usepackage{amsmath}

%% ---------------------------------------------------
%% Packages for URLs and hotlinks in the table of contents
%% and symbolic cross references using \ref
%% ---------------------------------------------------
\usepackage{hyperref}

%% ---------------------------------------------------
%% Packages for using HOL-generated macros and displays
%% ---------------------------------------------------
\usepackage{holtex}
\usepackage{holtexbasic}
% =====================================================================
%
% Macros for typesetting the HOL system manual
%
% =====================================================================

% ---------------------------------------------------------------------
% Abbreviations for words and phrases
% ---------------------------------------------------------------------

\newcommand\TUTORIAL{{\footnotesize\sl TUTORIAL}}
\newcommand\DESCRIPTION{{\footnotesize\sl DESCRIPTION}}
\newcommand\REFERENCE{{\footnotesize\sl REFERENCE}}
\newcommand\LOGIC{{\footnotesize\sl LOGIC}}
\newcommand\LIBRARIES{{\footnotesize\sl LIBRARIES}}
\usepackage{textcomp}

\newcommand{\bs}{\texttt{\char'134}} % backslash
\newcommand{\lb}{\texttt{\char'173}} % left brace
\newcommand{\rb}{\texttt{\char'175}} % right brace
\newcommand{\td}{\texttt{\char'176}} % tilde
\newcommand{\lt}{\texttt{\char'74}} % less than
\newcommand{\gt}{\texttt{\char'76}} % greater than
\newcommand{\dol}{\texttt{\char'44}} % dollar
\newcommand{\pipe}{\texttt{\char'174}}
\newcommand{\apost}{\texttt{\textquotesingle}}
% double back quotes ``
\newcommand{\dq}{\texttt{\char'140\char'140}}
%These macros were included by slind:

\newcommand{\holquote}[1]{\dq#1\dq}

\def\HOL{\textsc{Hol}}
\def\holn{\HOL}  % i.e. hol n(inety-eight), no digits in
                 % macro names is a bit of a pain; deciding to do away
                 % with hol98 nomenclature means that we just want to
                 % write HOL for hol98.
\def\holnversion{Kananaskis-11}
\def\holnsversion{Kananaskis~11} % version with space rather than hyphen
\def\LCF{{\small LCF}}
\def\LCFLSM{{\small LCF{\kern-.2em}{\normalsize\_}{\kern0.1em}LSM}}
\def\PPL{{\small PP}{\kern-.095em}$\lambda$}
\def\PPLAMBDA{{\small PPLAMBDA}}
\def\ML{{\small ML}}
\def\holmake{\texttt{Holmake}}

\newcommand\ie{\mbox{\textit{i{.}e{.}}}}
\newcommand\eg{\mbox{\textit{e{.}g{.}}}}
\newcommand\viz{\mbox{viz{.}}}
\newcommand\adhoc{\mbox{\it ad hoc}}
\newcommand\etal{{\it et al.\/}}
% NOTE: \etc produces wrong spacing if used between sentences, that is
% like here \etc End such sentences with non-macro etc.
\newcommand\etc{\mbox{\textit{etc{.}}}}

% ---------------------------------------------------------------------
% Simple abbreviations and macros for mathematical typesetting
% ---------------------------------------------------------------------

\newcommand\fun{{\to}}
\newcommand\prd{{\times}}

\newcommand\conj{\ \wedge\ }
\newcommand\disj{\ \vee\ }
\newcommand\imp{ \Rightarrow }
\newcommand\eqv{\ \equiv\ }
\newcommand\cond{\rightarrow}
\newcommand\vbar{\mid}
\newcommand\turn{\ \vdash\ } % FIXME: "\ " results in extra space
\newcommand\hilbert{\varepsilon}
\newcommand\eqdef{\ \equiv\ }

\newcommand\natnums{\mbox{${\sf N}\!\!\!\!{\sf N}$}}
\newcommand\bools{\mbox{${\sf T}\!\!\!\!{\sf T}$}}

\newcommand\p{$\prime$}
\newcommand\f{$\forall$\ }
\newcommand\e{$\exists$\ }

\newcommand\orr{$\vee$\ }
\newcommand\negg{$\neg$\ }

\newcommand\arrr{$\rightarrow$}
\newcommand\hex{$\sharp $}

\newcommand{\uquant}[1]{\forall #1.\ }
\newcommand{\equant}[1]{\exists #1.\ }
\newcommand{\hquant}[1]{\hilbert #1.\ }
\newcommand{\iquant}[1]{\exists ! #1.\ }
\newcommand{\lquant}[1]{\lambda #1.\ }

\newcommand{\leave}[1]{\\[#1]\noindent}
\newcommand\entails{\mbox{\rule{.3mm}{4mm}\rule[2mm]{.2in}{.3mm}}}

% ---------------------------------------------------------------------
% Font-changing commands
% ---------------------------------------------------------------------

\newcommand{\theory}[1]{\hbox{{\small\tt #1}}}
\newcommand{\theoryimp}[1]{\texttt{#1}}

\newcommand{\con}[1]{{\sf #1}}
\newcommand{\rul}[1]{{\tt #1}}
\newcommand{\ty}[1]{\textsl{#1}}

\newcommand{\ml}[1]{\mbox{{\def\_{\char'137}\texttt{#1}}}}
\newcommand{\holtxt}[1]{\ml{#1}}
\newcommand\ms{\tt}
\newcommand{\s}[1]{{\small #1}}

\newcommand{\pin}[1]{{\bf #1}}
% FIXME: for multichar symbols \mathit should be used.
\def\m#1{\mbox{\normalsize$#1$}}

% ---------------------------------------------------------------------
% Abbreviations for particular mathematical constants etc.
% ---------------------------------------------------------------------

\newcommand\T{\con{T}}
\newcommand\F{\con{F}}
\newcommand\OneOne{\con{One\_One}}
\newcommand\OntoSubset{\con{Onto\_Subset}}
\newcommand\Onto{\con{Onto}}
\newcommand\TyDef{\con{Type\_Definition}}
\newcommand\Inv{\con{Inv}}
\newcommand\com{\con{o}}
\newcommand\Id{\con{I}}
\newcommand\MkPair{\con{Mk\_Pair}}
\newcommand\IsPair{\con{Is\_Pair}}
\newcommand\Fst{\con{Fst}}
\newcommand\Snd{\con{Snd}}
\newcommand\Suc{\con{Suc}}
\newcommand\Nil{\con{Nil}}
\newcommand\Cons{\con{Cons}}
\newcommand\Hd{\con{Hd}}
\newcommand\Tl{\con{Tl}}
\newcommand\Null{\con{Null}}
\newcommand\ListPrimRec{\con{List\_Prim\_Rec}}


\newcommand\SimpRec{\con{Simp\_Rec}}
\newcommand\SimpRecRel{\con{Simp\_Rec\_Rel}}
\newcommand\SimpRecFun{\con{Simp\_Rec\_Fun}}
\newcommand\PrimRec{\con{Prim\_Rec}}
\newcommand\PrimRecRel{\con{Prim\_Rec\_Rel}}
\newcommand\PrimRecFun{\con{Prim\_Rec\_Fun}}

\newcommand\bool{\ty{bool}}
\newcommand\num{\ty{num}}
\newcommand\ind{\ty{ind}}
\newcommand\lst{\ty{list}}

% ---------------------------------------------------------------------
% \minipagewidth = \textwidth minus 1.02 em
% ---------------------------------------------------------------------

\newlength{\minipagewidth}
\setlength{\minipagewidth}{\textwidth}
\addtolength{\minipagewidth}{-1.02em}

% ---------------------------------------------------------------------
% Environment for the items on the title page of a case study
% ---------------------------------------------------------------------

\newenvironment{inset}[1]{\noindent{\large\bf #1}\begin{list}%
{}{\setlength{\leftmargin}{\parindent}%
\setlength{\topsep}{-.1in}}\item }{\end{list}\vskip .4in}

% ---------------------------------------------------------------------
% Macros for little HOL sessions displayed in boxes.
%
% Usage: (1) \setcounter{sessioncount}{1} resets the session counter
%
%        (2) \begin{session}\begin{verbatim}
%             .
%              < lines from hol session >
%             .
%            \end{verbatim}\end{session}
%
%            typesets the session in a numbered box.
% ---------------------------------------------------------------------

\newlength{\hsbw}
\setlength{\hsbw}{\textwidth}
\addtolength{\hsbw}{-\arrayrulewidth}
\addtolength{\hsbw}{-\tabcolsep}
\newcommand\HOLSpacing{13pt}

\newcounter{sessioncount}
\setcounter{sessioncount}{0}

\newenvironment{session}{\begin{flushleft}
 \refstepcounter{sessioncount}
 \begin{tabular}{@{}|c@{}|@{}}\hline
 \begin{minipage}[b]{\hsbw}
 \vspace*{-.5pt}
 \begin{flushright}
 \rule{0.01in}{.15in}\rule{0.3in}{0.01in}\hspace{-0.35in}
 \raisebox{0.04in}{\makebox[0.3in][c]{\footnotesize\sl \thesessioncount}}
 \end{flushright}
 \vspace*{-.55in}
 \begingroup\small\baselineskip\HOLSpacing}{\endgroup\end{minipage}\\ \hline
 \end{tabular}
 \end{flushleft}}

% ---------------------------------------------------------------------
% Macro for boxed ML functions, etc.
%
% Usage: (1) \begin{holboxed}\begin{verbatim}
%               .
%               < lines giving names and types of mk functions >
%               .
%            \end{verbatim}\end{holboxed}
%
%            typesets the given lines in a box.
%
%            Conventions: lines are left-aligned under the "g" of begin,
%            and used to highlight primary reference for the ml function(s)
%            that appear in the box.
% ---------------------------------------------------------------------

\newenvironment{holboxed}{\begin{flushleft}
  \begin{tabular}{@{}|c@{}|@{}}\hline
  \begin{minipage}[b]{\hsbw}
% \vspace*{-.55in}
  \vspace*{.06in}
  \begingroup\small\baselineskip\HOLSpacing}{\endgroup\end{minipage}\\ \hline
  \end{tabular}
  \end{flushleft}}

% ---------------------------------------------------------------------
% Macro for unboxed ML functions, etc.
%
% Usage: (1) \begin{hol}\begin{verbatim}
%               .
%               < lines giving names and types of mk functions >
%               .
%            \end{verbatim}\end{hol}
%
%            typesets the given lines exactly like {boxed}, except there's
%            no box.
%
%            Conventions: lines are left-aligned under the "g" of begin,
%            and used to display ML code in verbatim, left aligned.
% ---------------------------------------------------------------------

\newenvironment{hol}{\begin{flushleft}
 \begin{tabular}{c@{}@{}}
 \begin{minipage}[b]{\hsbw}
% \vspace*{-.55in}
 \vspace*{.06in}
 \begingroup\small\baselineskip\HOLSpacing}{\endgroup\end{minipage}\\
 \end{tabular}
 \end{flushleft}}

% ---------------------------------------------------------------------
% Emphatic brackets
% ---------------------------------------------------------------------

\newcommand\leb{\lbrack\!\lbrack}
\newcommand\reb{\rbrack\!\rbrack}


% ---------------------------------------------------------------------
% Quotations
% ---------------------------------------------------------------------


%These macros were included by ap; they are used in Chapters 9 and 10
%of the HOL DESCRIPTION

\newcommand{\inds}%standard infinite set
 {\mbox{\rm I}}

\newcommand{\ch}%standard choice function
 {\mbox{\rm ch}}

\newcommand{\den}[1]%denotational brackets
 {[\![#1]\!]}

\newcommand{\two}%standard 2-element set
 {\mbox{\rm 2}}


\begin{document}

%% --------------------------------------------------- the listings
%% parameter "language" is set to "ML"
%% ---------------------------------------------------
\lstset{language=ML}

\maketitle{}

\begin{abstract}

  This article is mainly about the HOL excercise. By learning the video given by the teacher, I have mastered the basic use of functions of ML and the expression of variables, including int, list and string, and learned Logic Semantics. At the same time, I used emacs to compile the report and write the source code, which made me more familiar with the syntax of LaTaX.
  \begin{itemize}
  \item Problem statement
  \item Relevant code
  \item Results from running ML code or HOL proofs
  \end{itemize}

  For each problem or exercise-oriented chapter in the main body of
  the report there is a corresponding chapter in the Appendix containing the
  source code in ML. This source code is not pasted into the Appendix,
  rather it is input directly from the source code file itself. This
  means changes in source code are easily captured in the report by
  recompiling the report in \LaTeX{}. \\


  We introduce the use of style files and packages. Specifically, we
  use:
  \begin{itemize}
  \item A style file for the course, \textit{634format.sty}
  \item The \textit{listings} package for displaying and inputting ML
    source code
  \item HOL style files and commands to display interactive ML/HOL
    sessions
  \end{itemize}
 Finally, we show how to:
  \begin{itemize}
  \item Easily generate a table of contents for the report
  \item Refer to chapter and section labels in our report
  \end{itemize}
\end{abstract}

\begin{acknowledgments}
  I did this project by myself. I tried to write the HOL program for the first time by studying the video given by the teacher.
\end{acknowledgments}

\tableofcontents{}

\chapter{Executive Summary}
\label{cha:executive-summary}

\textbf{All requirements for this project are satisfied.}
Specifically,
\begin{description}
\item[Report Contents] \ \\
  Our report has the following content:
  \begin{enumerate}[{}]
  \item Chapter~\ref{cha:executive-summary}: Executive Summary
  \item Chapter~\ref{cha:exercise-2.5.1}: Exercise 2.5.1
    \begin{enumerate}[{}]
    \item Section~\ref{sec:problem-statement}: Problem Statement
    \item Section~\ref{sec:test-cases}: Test Cases
    \item Section~\ref{sec:relevant-code}: Relevant Code
    \item Section~\ref{sec:execution-transc}: Execution Transcripts
    \end{enumerate}
  \item Chapter~\ref{cha:exercise-3.4.1}: Source Code for Sample Exercise
    \begin{enumerate}[{}]
    \item Section~\ref{sec:problem-statement-1}: Problem Statement
    \item Section~\ref{sec:relevant-code-1}: Relevant Code
    \item Section~\ref{sec:execution-transc-1}: Execution Transcripts
    \end{enumerate}
  \item Chapter~\ref{cha:exercise-3.4.2}: Exercise 3.4.2
    \begin{enumerate}[{}]
    \item Section~\ref{sec:problem-statement-2}: Problem Statement
    \item Section~\ref{sec:relevant-code-2}: Relevant Code
    \item Section~\ref{sec:execution-transc-2}: Execution Transcripts
    \end{enumerate}
  \item Appendix~\ref{cha:source-code-exercise}: Source Code for
    Exercise 2.5.1
  \item Appendix~\ref{cha:source-code-exercise-1}: Source Code for
    Exercise 3.4.1
  \item Appendix~\ref{cha:source-code-exercise-2}: Source Code for
    Exercise 3.4.2
  \end{enumerate}

\item[Reproducibility in ML and \LaTeX{}] \ \\
  The ML and \LaTeX{} source files compile with no errors.
\end{description}

\chapter{Exercise 2.5.1}
\label{cha:exercise-2.5.1}

\section{Problem Statement}
\label{sec:problem-statement}

\subsection{Function to implement}
\label{sec:function-implement}

We are required to execute the following function in ML:\\
\begin{align*}
  timesPlus \;x\;y \;=\; ( \;x\times y\;,\;x+y\;)
\end{align*}

\subsection{Test cases}
\label{sec:test-cases}
The required tests for \emph{timesPlus} are as follows:
\begin{lstlisting}[frame=TBLR](*gives frame as two lines tblr*)
(* Test Cases specified in the requirements *)

timesPlus 100   27; 
timesPlus  10   26; 
timesPlus   1   25; 
timesPlus   2   24;
timesPlus  30   23; 
timesPlus  50  200;
\end{lstlisting}

\section{Relevant Code}
\label{sec:relevant-code}

The following code runs the function definition using
\textit{fun} in ML, and \textit{currying}, i.e., defining functions
with multiple arguments as a sequence of functions. This supports
partial evaluation.

\lstset{frameround=tftf}%%make edge of the frame round. f=false t=true..these are to set which all corners you want rounded.
%% frame gives the frame or border. Uppercase=2 and lower case=1, t=top,
%% r=right, B=below, L=left
\begin{lstlisting}[frame=trbl]
  fun timesPlus x y = ( x * y , x + y )
\end{lstlisting}

\section{Execution Transcripts}
\label{sec:execution-transc}

\setcounter{sessioncount}{0}
\begin{session}
\begin{scriptsize}
\begin{verbatim}

> timesPlus 100 27;
val it = (2700, 127): int * int 
\end{verbatim}
\end{scriptsize}
\end{session}

\begin{session}
\begin{scriptsize}
\begin{verbatim}

> timesPlus 10 26;
val it = (260, 36): int * int 
\end{verbatim}
\end{scriptsize}
\end{session}

\begin{session}
\begin{scriptsize}
\begin{verbatim}

> timesPlus 1 25;
val it = (25, 26): int * int 
\end{verbatim}
\end{scriptsize}
\end{session}

\begin{session}
\begin{scriptsize}
\begin{verbatim}

> timesPlus 2 24;
val it = (48, 26): int * int 
\end{verbatim}
\end{scriptsize}
\end{session}

\begin{session}
\begin{scriptsize}
\begin{verbatim}

> timesPlus 30 23;
val it = (690, 53): int * int
\end{verbatim}
\end{scriptsize}
\end{session}

\begin{session}
\begin{scriptsize}
\begin{verbatim}

> timesPlus 50 200;
val it = (10000, 250): int * int 
\end{verbatim}
\end{scriptsize}
\end{session}

\chapter{Exercise 3.4.1}
\label{cha:exercise-3.4.1}

\section{Problem Statement}
\label{sec:problem-statement-1}
We are to devise ML expressions for the following tasks where we have 
values and require to assign them to variables as specified:
\begin{enumerate}[{1.}] %%[] specifies what to enumerate - alphabets or numbers
\item Devise a list of pairs [(0,''Alice''), (1,''Bob''),
  (3,''Carol''), (4,''Dan'')] and assign it the name listA
\item Using listA and pattern matching, create the following value
  assignments: elB has the value (0,''Alice'') and listB has the value
  [(1,''Bob''), (3,''Carol''), (4,''Dan'')]
\item Using elB, listB and pattern matching, create the following
  value assignments: elC1 has the value 0, elC2 has the value
  ``Alice'', elC3 has the value (1,''Bob''), elC4 has the value
  (3,''Carol''), and elC5 has the value (4,''Dan'').
\end{enumerate}

\section{Relevant Code}
\label{sec:relevant-code-1}

The following code uses value declarations and pattern
matching on tuples and lists.

\lstset{frameround=tftf}
\begin{lstlisting}[frame=trbl]
val listA = [(0,"Alice"),(1,"Bob"),(3,"Carol"),(4,"Dan")]; 
val (elB :: listB) = listA; 
val (elC1, elC2) = elB; 
val (elC3 :: (elC4 :: (elC5 :: []))) = listB;
\end{lstlisting}

\section{Execution Transcripts}
\label{sec:execution-transc-1}

\setcounter{sessioncount}{0}
\begin{session}
\begin{scriptsize}
\begin{verbatim}

> val listA = [(0,"Alice"), (1,"Bob"), (3,"Carol"), (4,"Dan")];
val listA = [(0, "Alice"), (1, "Bob"), (3, "Carol"), (4, "Dan")] : (int * string) list
\end{verbatim}
\end{scriptsize}
\end{session}

\begin{session}
\begin{scriptsize}
\begin{verbatim}

> val (elB :: listB) = listA;
val elB = (0, "Alice"): int * string
val listB = [(1, "Bob"), (3, "Carol"), (4, "Dan")] : (int * string) list
\end{verbatim}
\end{scriptsize}
\end{session}

\begin{session}
\begin{scriptsize}
\begin{verbatim}

> val (elC1, elC2) = elB;
val elC1 = 0: int
val elC2 = "Alice": string
> val (elC3 :: (elC4 :: (elC5 :: []))) = listB;
val elC3 = (1, "Bob"): int * string
val elC4 = (3, "Carol"): int * string
val elC5 = (4, "Dan"): int * string
\end{verbatim}
\end{scriptsize}
\end{session}

\chapter{Exercise 3.4.2}
\label{cha:exercise-3.4.2}

\section{Problem Statement}
\label{sec:problem-statement-2}

We are to evaluate the following assignments in the
order in which they appear in HOL, explain the errors that HOL
detects using comments and store the results in ex-3-4-2.trans file:\\
\begin{enumerate}[{1.}]
\item val (x1, x2, x3) = (1, true, "Alice");
\item val pair1 = (x1, x3);
\item val list1 = [0, x1, 2];
\item val list2 = [x2, x1];
\item val list3 = (1 :: [x3]);
\end{enumerate}

\section{Relevant Code}
\label{sec:relevant-code-2}

The following code uses value declarations and pattern
matching on tuples. Lists are given as part of the problem statement

\lstset{frameround=tftf}
\begin{lstlisting}[frame=trbl]
val (x1, x2, x3) = (1, true, "Alice");
val pair1 = (x1, x3);
val list1 = [0, x1, 2];
val list2 = [x2, x1];
val list3 = (1 :: [x3]);
\end{lstlisting}

\section{Execution Transcripts}
\label{sec:execution-transc-2}

\setcounter{sessioncount}{0}
\begin{session}
\begin{scriptsize}
\begin{verbatim}

> val (x1, x2, x3) = (1, true, "Alice");
val x1 = 1: int
val x2 = true: bool
val x3 = "Alice": string
\end{verbatim}
\end{scriptsize}
\end{session}

\begin{session}
\begin{scriptsize}
\begin{verbatim}

> val pair1 = (x1, x3);
val pair1 = (1, "Alice"): int * string
\end{verbatim}
\end{scriptsize}
\end{session}

\begin{session}
\begin{scriptsize}
\begin{verbatim}

> val list1 = [0, x1, 2];
val list1 = [0, 1, 2]: int list
\end{verbatim}
\end{scriptsize}
\end{session}

\begin{session}
\begin{scriptsize}
\begin{verbatim}

> val list2 = [x2, x1];
poly: : error: Elements in a list have different types.
   Item 1: x2 : bool
   Item 2: x1 : int
   Reason:
      Can't unify bool (*In Basis*) with int (*In Basis*)
         (Different type constructors)
Found near [x2, x1]
Static Errors
\end{verbatim}
\end{scriptsize}
\end{session}

\begin{session}
\begin{scriptsize}
\begin{verbatim}

> val list3 = (1 :: [x3]);
poly: : error: Type error in function application.
   Function: :: : int * int list -> int list
   Argument: (1, [x3]) : int * string list
   Reason:
      Can't unify int (*In Basis*) with string (*In Basis*)
         (Different type constructors)
Found near (1 :: [x3])
Static Errors
\end{verbatim}
\end{scriptsize}
\end{session}


%% ------------------------------------------
%% Change to letters for appendix
%% ------------------------------------------

%% ------------------------------------------
%% this restarts the section numbering
%% ------------------------------------------
\appendix{} 


%% ------------------------------------------
% label using capital letters
%% ------------------------------------------
\renewcommand{\thechapter}{\Alph{chapter}}

\chapter{Source Code for Exercise 2.5.1}
\label{cha:source-code-exercise}

The following code is from \textit{ex-2-5-1.sml}
\lstinputlisting{../ML/ex-2-5-1.sml}

\chapter{Source Code for Exercise 3.4.1}
\label{cha:source-code-exercise-1}

The following code is from \textit{ex-3-4-1.sml}
\lstinputlisting{../ML/ex-3-4-1.sml}

\chapter{Source Code for Exercise 3.4.2}
\label{cha:source-code-exercise-2}

The following code is from \textit{ex-3-4-2.sml}
\lstinputlisting{../ML/ex-3-4-2.sml}

\end{document}